\documentclass[11pt,a4paper]{article}

\usepackage[headsep=1cm,headheight=3cm,left=3.5cm,right=3.5cm,top=2.5cm,bottom=2.5cm,a4paper]{geometry}

\linespread{1.3}
\setlength{\parindent}{0pt}
\setlength{\parskip}{1em}

\usepackage[spanish]{babel}
\usepackage[utf8]{inputenc}

% FUENTES
\usepackage[sfdefault]{FiraSans} %% option 'sfdefault' activates Fira Sans as the default text font
\usepackage{FiraMono}
\usepackage[T1]{fontenc}
\renewcommand*\oldstylenums[1]{{\firaoldstyle #1}}


\usepackage{enumitem}
\setlist[itemize]{leftmargin=*}
\setlist[enumerate]{leftmargin=*}

\usepackage{tabularx}
\usepackage{adjustbox}

\usepackage{xcolor}

\definecolor{nord0}{HTML}{2e3440}
\definecolor{nord4}{HTML}{d8dee9}
\definecolor{nord11}{HTML}{bf616a}
\definecolor{nord10}{HTML}{5e81ac}
\definecolor{nord11}{HTML}{bf616a}

\usepackage{listings}

\lstset{
language=C++,
numbers=left,
numberstyle=\small\ttfamily,
basicstyle=\small\ttfamily\color{nord0},
otherkeywords={self},             % Add keywords here
keywordstyle=\color{nord10},
emphstyle=\color{nord11},    % Custom highlighting style
stringstyle=\color{nord11},
showstringspaces=false,            %
breaklines=true,
postbreak=\mbox{\textcolor{nord11}{$\hookrightarrow$}\space},
}

\title{Inteligencia Artificial. Memoria práctica 2 \\ \Large{Los mundos de Belkan}}
\author{Antonio Coín Castro. 3º DGIIM}
\date{}

\begin{document}

\maketitle

\section*{Consideraciones previas}%
\label{sec:consideraciones_previas}

Quiero aprovechar para mostrar mi rechazo al sistema de evaluación de las prácticas 2 y 3 de esta asignatura, en concreto, al aspecto competitivo del mismo.

La evaluación debe basarse en la comprobación de la adquisición de las competencias que corresponden a cada práctica, no en averiguar qué estudiante puede superar al resto de sus compañeros. Esta forma de evaluar rompe además las dinámicas de grupo de colaboración y compañerismo (algo en lo que también influye la insistencia constante sobre el carácter individual de las prácticas y el plagio). 

\textbf{La educación no es una competición.} 

\section*{Nivel 1}

Para este nivel, la única dificultad mas allá de las consideraciones particulares del código proporcionado, es definir un algoritmo para ir de un origen a un destino, pasando únicamente por casillas permitidas. Elegimos el algoritmo A* visto en clase, que además nos garantiza que el camino encontrado es óptimo.

El algoritmo A* utiliza una lista de \textbf{abiertos} para los nodos que le quedan por expandir, y una lista de \textbf{cerrados} para los nodos ya expandidos. La lista de abiertos está ordenada buscando el valor más pequeño de \textit{fScore} en cada nodo, que no es otra cosa que la suma del coste actual para ir del origen a ese nodo sumado con la distancia Manhattan al destino. En cada paso, se coge el nodo más prometedor de abiertos, se expanden sus vecinos permitidos, y se ve si mejoran a algún nodo que ya estaba en abiertos, actualizando en consecuencia.

La estructura de datos utilizada es la clase \textbf{node}, que no es más que una casilla junto con un puntero a su casilla `'padre'' en el algoritmo, y una lista de acciones para llegar desde el padre hasta él. De esta forma se puede reconstruir el camino.

\section*{Nivel 2}

Es casi igual al nivel 1, salvo que si ve que al avanzar hacia delante hay un aldeano, se espera quieto hasta que se vaya de esa casilla. No se recalcula la ruta, pues es mucho más probable que un aldeano avance y se quite de la casilla a que solo gire o se quede quieto.

\section*{Nivel 3}


\underline{Antes de conocer tu posición} \\

Lo que hacemos es una especie de `'mapa de calor'' considerando antes las casillas por las que hace más tiempo que no nos movemos, con la esperanza de ir recorriendo el mapa y encontrar una K a través de los sensores. Como particularidad, al empezar en el mapa damos una vuelta completa, para ver aunque sea al principio todo a nuestro alrededor.

\underline{Una vez conocida tu posición} \\

En este caso, simplemente vamos haciendo A* hacia el destino, con dos consideraciones:
\begin{itemize}
	\item Las casillas marcadas con '?' son transitables en principio.
	\item Si ejecutando el plan llegamos a una casilla que en realidad no es transitable, se recalcula el camino con la nueva información. Nunca se choca ni se muere.
\end{itemize}

\end{document}
